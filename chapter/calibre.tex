% !TEX encoding = UTF-8
% !TEX program = pdflatex
% !TEX root = ../guidaebooklatex.tex


\chapter{calibre}

Abbiamo ottenuto un file html ``intermedio'', che deve ancora
essere processato da uno strumento adatto a convertire il nostro testo in 
mobi/epub. Fa quindi al caso nostro il programma calibre, ed in particolare la 
sua versione a riga di comando ebook-convert. Anche in questo caso dovremo 
passare all'eseguibile alcune opzioni, per corredare il nostro ebook di una 
serie di informazioni:

\begin{lstlisting}
ebook-convert \
  tmp/main_semplice.html \
  output.epub \
  --output-profile=kindle  \
  --change-justification justify \
  --max-toc-links 0 \
  --authors="Pietro Giuffrida"  \
  --title="Da LaTeX ad ebook in tre semplici mosse"  \
  --publisher="Centro Internazionale per la Ricerca Filosofica"  \
  --series="Epekeina. International Journal of Ontology. History and Critics" \
  --pubdate="2012-11-10"  \
  --cover="cover.png"

\end{lstlisting}

Vediamo nel dettaglio il significato delle opzioni meno esplicite: 
\begin{itemize}
\item[--output-profile=kindle]
\item[--change-justification justify]
\item[--max-toc-links 0]
\item[--authors="Pietro Giuffrida"]
\item[--title="Da LaTeX ad ebook in tre semplici mosse"]
\item[--publisher="Guide del GuIT"]
\item[--pubdate="2013-03-29"]
\item[--cover="cover.png"]
l'opzione cover permette di specificare un'immagine da
includere nell'ebook. Consiglio di non trascurare tale opzione, perché gli 
ebook-reader usano una miniatura della copertina per rappresentare la libreria, 
ed in sua assenza usano una miniatura della prima pagina del testo, con il 
risultato di rendere irriconoscibile il libro e sgradevole il suo aspetto. 
\end{itemize}

\chapter{Tutto con un solo script}
Chiaramente i passaggi necessari per la produzione di un ebook dai sorgenti 
\LaTeX{} possono essere facilmente integrati in un unico script, per evitare di
dover fare tutti i passaggi a mano e di dover scrivere interminabili istruzioni 
sulla riga di comando. A questo scopo può sicuramente essere apprestato un 
makefile, ma per il momento io mi sono limitato ad un semplice script bash, più 
flessibile in fase di sperimentazione e, almeno per le mie competenze, più 
facilmente implementabile con le istruzioni sed per commentare il codice 
indesiderato nel file tex. Un esempio di tale script potrebbe essere questo:

\begin{lstlisting}
sed -e "s/\\\looseness=+1//g" \
    -e "s/\\\looseness=-1//g" \
    main_semplice.tex > epub.tex
\end{lstlisting}
