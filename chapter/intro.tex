% !TEX encoding = UTF-8
% !TEX program = pdflatex
% !TEX root = ../guidaebooklatex.tex


\chapter{Introduzione}

Un metodo per ottenere dal sorgente \LaTeX, oltre al normale output pdf, anche un 
output compatibile con i più diffusi e-reader in commercio, ovvero epub o mobi. 
Questi ultimi due formati sono in realtà dei dialetti html, i cui sorgenti sono 
compressi in un file zippato, appunto con estensione mobi o eub. La guida non 
scende, al momento, nel dettaglio di tali formati e delle possibili 
personalizzazioni. L'onere della conversione definitva è infatti lasciata a dei 
programmi esterni alla famiglia tex, ovvero calibre. Il flusso di lavoro
prevede l'aggiunta di due passaggi alla normale stesura di un documento con 
Latex. Terminata la stesura del documento occorrerà infatti seguire i seguenti passaggi.

\begin{comment}
\begin{center}
\begin{forest}
[latex
[bibtex
[latex
[latex
[latex2html
[calibre]]]]]]
\end{forest}
\end{center}
\end{comment}

I primi passaggi sono noti a qualsiasi utente Latex. Si tratta di generare il 
documento in dvi (e non in pdf) con compilando il sorgente tex con l'eseguibile 
latex, aggiungendo poi i passaggi necessari alla generazione della bibliografia 
con lo storico pacchetto bibtex (e non con biblatex!).

Il programma latex2html dovrà essere adeguatamente configurato in modo da
produrre
un output piuttosto scarno, privo di elementi che potrebbero essere mal 
interpretati da calibre nella generazione del ebook definitivo. Per ottenere un 
risultato accettabile bisogna considerare che latex2html riconosce un numero
estremamente limitato di pacchetti, nel cui numero non è possibile neppure 
vantare ifthenesle. Di conseguenza in primo problema cui ovviare è proprio 
quello di ottenere un main file \LaTeX compatibile con latex2html.


