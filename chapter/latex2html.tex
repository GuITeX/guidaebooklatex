% !TEX encoding = UTF-8
% !TEX program = pdflatex
% !TEX root = ../guidaebooklatex.tex


\chapter{Un preambolo per latex2html}

Ottenere un preambolo adeguato alla compilazione con latex2html è un'operazione
di per sé che non offre particolari difficoltà. È infatti sufficiente includere
solo e soltanto i pacchetti strettamente indispensabili. Chiaramente però un
preambolo di tale tipo limita fortemente le opzioni normalmente a disposizione
di un utente Latex. Nell'approccio che intendo suggerire ho quindi scelto di
utilizzare contemporaneamente due main file distinti, uno destinato alla
compilazione con pdflatex, e l'altro alla produzione del ebook. In entrambi i
main file non dovrà quindi esser presente alcuna parte del testo, ma soltanto le
chiamate ai vari pacchetti ed i comandi input necessari a caricare i vari file
tex contenenti il testo vero e proprio. Quest'approccio ha chiaramente la
controindicazione fondamentale di dover utilizzare nella stesura del testo solo
e soltanto comandi compatibili con il preambolo scarno progettato per
latex2html. In altri termini le uniche vere differenze tra i due preamboli
possono consistere in ridefinizioni di ambienti e pacchetti standard,
normalmente interpretati senza bisogno di pacchetti aggiuntivi, ed opzioni per
la formattazione del testo nei suoi vari aspetti (font, geometria della pagina
etc.) comunque irrilevanti, se non dannosi, per la produzione del file
html.\footnote
  {Purtroppo latex2html non riconosce neppure i comandi generalmente
  utilizzati per la revisione finale del testo, quali enlargethispage. A questo
  proposito si veda lo script in appendice}

Un esempio minimo funzionante potrebbe essere il seguente.

\begin{lstlisting}
\documentclass[a4paper,oneside,10pt]{article}
\usepackage[utf8x]{inputenc}
\usepackage[T1]{fontenc}
\usepackage[italian]{babel}

\usepackage{natbib}
\bibpunct[, ]{}{}{;}{a}{,}{,}

\begin{document}
\pagestyle{empty}
\bibliographystyle{abbrvnat}

\input{ilmiotesto}

\bibliography{bibfile}

\end{document}
\end{lstlisting}

Qualche commento su queste poche righe di codice:
\begin{itemize}
\item
non ci sono indicazioni di font o di geometria della pagina. Sarebbe
superfluo e probabilmente otterremmo degli errori in fase di compilazione

\item
uso il pacchetto natbib, e non il più moderno biblatex. Ciò
perché quest'ultimo non è compatibile con latex2html.
\end{itemize}

Nel main file più complesso, destinato alla produzione del pdf, potremo
comunque usare  il pacchetto biblatex, attenendoci però in fase di
elaborazione della base dati bib a quei campi interpretabili anche dal più
vecchio bibtex. Giusto per fare un esempio, non dovremo usare il campo location,
ma quello xxxx. Nel caso dei campi tipici del pacchetto biblatex-philosophy
potremo invece usare i campi orig-x ma essi saranno ignorati.

Ciò detto, appena in grado di compilare senza errori il sorgente tex nella sua
versione  scarna potremo passare alla fase successiva.

\section{latex2html}

Per ottenere il file html dal nostro file tex dovremo compilare quest'ultimo 
mediante l'eseguibile latex2html. Tale programma però, salvo specificare alcune
opzioni di compilazione, produce un file html per ciascuna section, e dota 
ciascuna pagina di una serie di strumenti di lettura utili e sensati solo se 
destinati all'uso in un browser tradizionale, ma del tutto inutili nel nostro 
caso. Nella mia esperienza attuale il programma dovrà quindi essere utilizzato 
nel modo seguente:


\begin{lstlisting}
latex2html -html_version 4.0,latin1,unicode \
 -nolatex \
 -no_footnode  \
 -split 0  \
 -no_navigation  \
 -info 0  \
 -mkdir  \
 -dir tmp  \
 main_semplice.tex
\end{lstlisting}

Vediamo il significato delle opzioni utilizzate:

\begin{description}
\item[-nolatex] a

\item[-not\_footnode] b

\item[-split 0] c

\item[-no\_navigation] d

\item[-info 0] e

\item[-mkdir] f

\item[-dir tmp] g

\end{description}


Chiaramente potere aprire il file html così generato con qualsiasi browser, in 
modo anche da potervi rendere conto di eventuali errori. Un problema che ho 
riscontrato in questa fase è relativo all'uso dei comandi spesso utilizzati per 
la revisione finale del testo, in particolare enlargethispage e compagnia
bella. 
Tali comandi non vengono riconosciuti da latex2html, che li riporta in modo
letterale nel file html definitivo. Di conseguenza se non interveniamo anche il 
nostro ebook vanterà le misteriose scritte enlargethispage all'interno del 
testo! Per ovviare a questo problema possiamo procedere in diversi modi: 
\begin{itemize}
\item a mano, ovvero commentando i comandi suddetti prima di 
compilare con \LaTeX e con latex2html, ripristinando poi la situazione iniziale
non appena avremo terminato tale fase.

\item in modo automatico, chiedendo ad latex2html di ignorare tali comandi. Solo
che non ci sono mai riuscito.

\item in modo semiautomatico, ovvero affidandoci ad uno script che adatti il 
testo alle nostre esigenze.
\end{itemize}

Per il momento io seguo questa terza opzione, lanciando già il comando
latex2html tramite uno script che si occupa preventivamente di commentare tutte
le occorrenze dei comandi indesiderati, di eseguire latex2html, e poi di
rimuovere i commenti applicati.

\begin{lstlisting}
\end{lstlisting}

